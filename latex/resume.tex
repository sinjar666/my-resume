%%%%%%%%%%%%%%%%%
% This is an sample CV template created using altacv.cls
% (v1.6.5, 3 Nov 2022) written by LianTze Lim (liantze@gmail.com), based on the
% CV created by BusinessInsider at http://www.businessinsider.my/a-sample-resume-for-marissa-mayer-2016-7/?r=US&IR=T
%
%% It may be distributed and/or modified under the
%% conditions of the LaTeX Project Public License, either version 1.3
%% of this license or (at your option) any later version.
%% The latest version of this license is in
%%    http://www.latex-project.org/lppl.txt
%% and version 1.3 or later is part of all distributions of LaTeX
%% version 2003/12/01 or later.
%%%%%%%%%%%%%%%%

%% Use the "normalphoto" option if you want a normal photo instead of cropped to a circle
% \documentclass[10pt,a4paper,normalphoto]{altacv}

\documentclass[10pt,a4paper,ragged2e,withhyper]{altacv}

\usepackage{xifthen}
\newboolean{TWO_COLUMN_LAYOUT}
\newboolean{ENABLE_GRAPHICS}

\setboolean{TWO_COLUMN_LAYOUT}{true}
\setboolean{ENABLE_GRAPHICS}{true}

\newif\ifshowIcons
%\showIconstrue    % or \showIconsfalse to hide icons
\showIconsfalse

% Converts string like "ChartLine" to \faChartLine
\newcommand{\faicon}[1]{\csname fa#1\endcsname}

% Conditionally include the icon based on flag
\newcommand{\maybeicon}[1]{%
  \ifshowIcons
    \faicon{#1}%
  \fi
}

\newcommand{\maybeIconOrLabel}[2]{%
  \ifshowIcons
    \maybeicon{#1}%
  \else
    \textbf{#2}%
  \fi
}
\newcommand{\resdataName}{Srijan Mukherjee}
\newcommand{\resdataTagline}{Hands-on Engineering Leader \& Problem Solver}
\newcommand{\resdataEmail}{srijmukh070@gmail.com}
\newcommand{\resdataPhone}{+91-8660685187}
\newcommand{\resdataLocation}{Bengaluru, KA}
\newcommand{\resdataGithubHandle}{sinjar666}
\newcommand{\resdataLinkedinHandle}{sinjar666}
\newcommand{\yearsExp}{13}



%% AltaCV uses the fontawesome5 package.
%% See http://texdoc.net/pkg/fontawesome5 for full list of symbols.

% Change the page layout if you need to
\geometry{left=1.25cm,right=1.25cm,top=1.5cm,bottom=1.5cm,columnsep=1.2cm}

\ifthenelse{\boolean{TWO_COLUMN_LAYOUT}} {
  % The paracol package lets you typeset columns of text in parallel
  \usepackage{paracol}
}{}

\usepackage{hyphenat}
\usepackage[english]{babel}

% Change the font if you want to, depending on whether
% you're using pdflatex or xelatex/lualatex
\ifxetexorluatex
  % If using xelatex or lualatex:
  \setmainfont{Lato}
\else
  % If using pdflatex:
  \usepackage[default]{lato}
\fi

% Change the colours if you want to
\definecolor{ResumeBrown}{HTML}{663D14}
\definecolor{SlateGrey}{HTML}{2E2E2E}
\definecolor{LightGrey}{HTML}{666666}
% \colorlet{name}{black}
\colorlet{tagline}{ResumeBrown}
\colorlet{heading}{ResumeBrown}
\colorlet{headingrule}{ResumeBrown}
% \colorlet{subheading}{PastelRed}
\colorlet{accent}{ResumeBrown}
\colorlet{emphasis}{SlateGrey}
\colorlet{body}{LightGrey}

% Change some fonts, if necessary
% \renewcommand{\namefont}{\Huge\rmfamily\bfseries}
% \renewcommand{\personalinfofont}{\footnotesize}
% \renewcommand{\cvsectionfont}{\LARGE\rmfamily\bfseries}
% \renewcommand{\cvsubsectionfont}{\large\bfseries}

% Change the bullets for itemize and rating marker
% for \cvskill if you want to
\renewcommand{\itemmarker}{{\small\textbullet}}
\renewcommand{\ratingmarker}{\maybeicon{Circle}}

% Final rendering command: icon (if enabled), title, description
\newcommand{\achievementEntry}[3]{%
  \cvachievement{\maybeicon{#1}}{\textbf{#2}}{#3}%
  \divider
}

%% Use (and optionally edit if necessary) this .tex if you
%% want to use an author-year reference style like APA(6)
%% for your publication list
% \input{pubs-authoryear}

%% Use (and optionally edit if necessary) this .tex if you
%% want an originally numerical reference style like IEEE
%% for your publication list
\input{pubs-num}

%% sample.bib contains your publications
\addbibresource{sample.bib}

\begin{document}
\name{\resdataName}
\tagline{\resdataTagline}
%% You can add multiple photos on the left or right
\photoR{2.5cm}{resume-img}
% \photoL{2cm}{Yacht_High,Suitcase_High}
\personalinfo{%
  % Not all of these are required!
  % You can add your own with \printinfo{symbol}{detail}
  \email{\resdataEmail}
  \phone{\resdataPhone}
  %\mailaddress{Address, Street, 00000 County}
  \location{\resdataLocation}
  \github{\resdataGithubHandle}
  \linkedin{\resdataLinkedinHandle}

  %% You can add your own arbitrary detail with
  %% \printinfo{symbol}{detail}[optional hyperlink prefix]
  % \printinfo{\maybeicon{Paw}}{Hey ho!}
  %% Or you can declare your own field with
  %% \NewInfoFiled{fieldname}{symbol}[optional hyperlink prefix] and use it:
  % \NewInfoField{gitlab}{\maybeicon{Gitlab}}[https://gitlab.com/]
  % \gitlab{your_id}
	%%
  %% For services and platforms like Mastodon where there isn't a
  %% straightforward relation between the user ID/nickname and the hyperlink,
  %% you can use \printinfo directly e.g.
  % \printinfo{\maybeicon{Mastodon}}{@username@instace}[https://instance.url/@username]
  %% But if you absolutely want to create new dedicated info fields for
  %% such platforms, then use \NewInfoField* with a star:
  % \NewInfoField*{mastodon}{\maybeicon{Mastodon}}
  %% then you can use \mastodon, with TWO arguments where the 2nd argument is
  %% the full hyperlink.
  % \mastodon{@username@instance}{https://instance.url/@username}
}

\makecvheader
Dynamic and results-oriented engineering leader with over \textbf{\yearsExp \ years} of experience in software development and team management. Proven track record in driving product innovation, optimizing engineering processes, and enhancing team performance. 
Seeking a Senior Engineering Leadership role to leverage expertise in distributed computing, agile methodologies, and cross-functional collaboration to deliver impactful solutions and foster a culture of excellence within the team


%% Depending on your tastes, you may want to make fonts of itemize environments slightly smaller
\AtBeginEnvironment{itemize}{\small}

\ifthenelse{\boolean{TWO_COLUMN_LAYOUT}} {
  %% Set the left/right column width ratio to 6:4.
  \columnratio{0.6}

  % Start a 2-column paracol. Both the left and right columns will automatically
  % break across pages if things get too long.
  \begin{paracol}{2}

  % Manually apply the body color to both columns
  \color{body}  % Apply the color at the beginning of left column
}{}

\cvsection{Experience}

\cvevent{Director, Software Development}{VISA Cloud Platform IaaS}{September 2023 -- Present}{Bengaluru, IN}
\begin{itemize}
    \item Re-architecture of Distributed Execution Engine, moving from an Active-Passive to an \textbf{Active-Active dual-writer} AP/CP tolerant hybrid with an in-built Disaster Recovery state machine.
    \item Led the development of an \textbf{MCP} based AI \textbf{agent} for taking actions from meeting transcripts. Increased accuracy with structured data using a modified \textbf{ReAct} prompting technique. 
    \item Delivered \textbf{Zero Downtime MongoDB to MySQL migration at 2000+ QPS} and the development of a highly available event messaging framework.
    \item Initiated the development of "vPlay," a cloud-based in-browser IDE aimed at reimagining the developer experience and mitigating left-shift inertia, with projected annual savings exceeding \textbf{\$1M}.
    \item Established and led a high-performing Bangalore Cloud Platform team of \textbf{18 members} (2 Managers + 16 Engineers), fostering collaboration and innovation.
    \item Oversaw a robust portfolio consisting of \textbf{18 microservices and frontends} and 5 SDKs, achieving year-over-year cost savings of approximately \textbf{\(\sim\)\$200M}.
    \item Strategically positioned VISA's Enterprise Cloud to evolve into a \textbf{SaaS offering} as part of a comprehensive 5-year growth plan.
    \item Championed the elevation of \textbf{Generative AI} capabilities and awareness, resulting in a \textbf{10\% increase in developer productivity} across the VISA Bengaluru Technology Center in Q1-Q2 FY-24.
    \item Led multiple site-level initiatives and actively mentored 3 employees across various teams and functions, enhancing the leadership quotient and overall technological competence of the Bengaluru Technology Center.
  \end{itemize}

\divider  

\cvevent{Senior Engineering Manager}{VISA Cloud Platform IaaS}{August 2021 -- August 2023}{Bengaluru, IN}
\begin{itemize}
    \item Launched \textbf{2 new products} in FY22-23 by repurposing \textbf{existing resources}, achieving delivery in early FY-24.
    \item Developed a team of \(\sim\)11 full-stack engineers driving design and development for \textbf{10+ services} automating day-2 lifecycle aspects of cloud infrastructure.
    \item Increased adoption of VISA's Enterprise Server Configuration Management Solution by \textbf{400\%} from FY-21 to FY-23.
    \item Reduced \textbf{hotfix frequency by 45\%} for critical products, achieving success rates >\textbf{98\%} and SLA adherence >\textbf{95\%}.
    \item Contributed \textbf{10-15\% of code reviews} for oversight and burst capacity.
    \item Managed project schedules and cost estimations for the portfolio.
    \item Co-authored \textbf{Intellectual Property} with the team.
  \end{itemize}

\divider  

\cvevent{Staff Full Stack Developer}{VISA Cloud Platform}{Jan 2019 -- July 2021}{Bengaluru, IN}
\begin{itemize}
    \item Key owner of the Core UI Platform for CloudView, the portal for VISA’s infrastructure provisioning and lifecycle management.
    \item Led VISA’s \textbf{hybrid cloud} initiative, focusing on \textbf{ESB and Enterprise Gateway Automation} and provisioning services.
    \item Introduced \textbf{HTTP/2 'SPDY'} streams and \textbf{Server-Sent Events} as alternatives to REST, driving server-side implementation.
    \item Mentored junior team members in enterprise coding practices and software architecture.
\end{itemize}

\divider  

\cvevent{Senior Full Stack Developer}{VISA Inc}{Feb 2017 -- Jan 2019}{Bengaluru, IN}
\begin{itemize}
    \item Developed essential features for \textbf{self-service} capabilities in cloud provisioning, focusing on \textbf{Kubernetes Container Orchestration}.
    \item Led UI development across components, ensuring a cohesive user experience.
    \item Enforced enterprise design patterns in UI development, including \textbf{Redux} and \textbf{RxJS}, promoting testable components.
    \item Designed an enterprise rapid application framework using Node.js, Express, and TypeScript, accelerating API development.
    \item Created an in-house \textbf{SPA agnostic micro-frontend framework} for internal app contributions.
    \item Established scalable \textbf{build and deploy strategies} for Node.js applications.
\end{itemize}

\divider  

\cvevent{From Project Trainee to Senior Engineer}{Honeywell Technology Solutions}{Dec 2012 -- Jan 2017}{Bengaluru, IN}
\begin{itemize}
    \item Developed software tools for productivity and process adherence in \textbf{Flight Management Systems} (FMS) Software.
    \item Created a web-based port of a legacy model-based UI with ARINC-661 widgets using AngularJS, Bootstrap, and Ionic.
    \item Engineered asynchronous network services for the Build Server with C\# .NET.
    \item Developed a proprietary \textbf{desktop application} for software requirements management using C\# .NET and WPF.
    \item Contributed to the UI and Display Proxy layers of \textbf{NG-FMS} deployed on major aircraft by \textbf{Boeing, Airbus, Dassault, and Gulfstream}.
\end{itemize}

\divider  

\cvevent{Summer Intern}{SRM Research Institute}{May 2012 -- June 2012}{Bengaluru, IN}
Developed "CLASSICO", a desktop application enabling one-to-one audio/video communication between rural and urban schools. Led the entire product development under tight deadlines and budget constraints.

\textbf{Technology Stack:} Python (Ubuntu/Debian), GStreamer, GTK+

\ifthenelse{\boolean{TWO_COLUMN_LAYOUT}} {
  %% Switch to the right column. This will now automatically move to the second
  %% page if the content is too long.
  \switchcolumn
  \color{body}  % Apply the color at the beginning of left column
}{}

\cvsection{Publications}

\nocite{*}

\printbibliography[heading=pubtype,title={\printinfo{\maybeicon{File}*[regular]}{Patents and Defensive Publications}}, type=patent]


\cvsection{Skills}
\cvtag{Private Cloud}
\cvtag{PaaS}
\cvtag{IaaS}
\cvtag{SaaS}

\divider\smallskip

\cvtag{Engineering Leadership}
\cvtag{Team Collaboration}
\cvtag{People Management}
\cvtag{Project Management}
\cvtag{Mentorship}
\cvtag{Adaptability}
\cvtag{Strategic Thinking}
\cvtag{Conflict Resolution}

\divider\smallskip

\cvtag{Distributed Computing}
\cvtag{NodeJS}
\cvtag{Java}
\cvtag{Spring}
\cvtag{Angular}
\cvtag{Linux}
\cvtag{Kafka}

\cvtag{Multithreading}
\cvtag{HTML/CSS}
\cvtag{C++}
\cvtag{C\# .NET}
\cvtag{HTTP/REST}
\cvtag{Virtualization}
\cvtag{MySQL}
\cvtag{MongoDB}
\cvtag{ElasticSearch}
\cvtag{Hazelcast}
\cvtag{GoLang}
\cvtag{Qt}
\cvtag{Erlang}
\cvtag{Software Design and Architecture}

\divider\smallskip

\cvtag{Git}
\cvtag{JIRA}
\cvtag{Splunk}
\cvtag{MS-Office}
\cvtag{K8s}


% Render Section
\cvsection{Most Proud Of}
% achievements-data.tex
\newcommand{\achievementItem}[3]{%
  % #1 = Icon name (string), #2 = Title, #3 = Description
  \achievementEntry{#1}{#2}{#3}%
}

\newcommand{\loadAchievements}{%
  \achievementItem{ChartLine}{Revolutionized Product Maturity}{Elevated automation runs from hundreds annually to an astounding \textbf{over 10 million} in FY-23-24, showcasing a transformative leap in efficiency.}
  \achievementItem{Check}{Unmatched Manager Effectiveness}{Achieved a \textbf{100\% manager effectiveness score} in the FY-23-24 anonymous employee survey, reflecting exceptional leadership and team cohesion.}
  \achievementItem{Trophy}{Portfolio Expansion}{Doubled the India team's portfolio by \textbf{2x in just 2 years} while maintaining the same number of engineers, demonstrating strategic resource management and innovation.}
  \achievementItem{Lightbulb}{Embracing a Growth Mindset}{Cultivated a proactive approach to change and continuous learning, enabling seamless transitions across diverse roles, technologies, and domains throughout my career.}
}

\loadAchievements

\cvsection{Education}

\cvevent{B. Tech\ in Computer Science}{SRM University}{2009 -- 2013}{Chennai IN}
\begin{itemize}
  \item \textbf{"ISBN Keeper"} (\textbf{Nokia OVI store}): A tool for tracking books by capturing ISBN barcodes, fetching data from the internet, and saving it.
  \item \textbf{"My Sehat"} (\textbf{Nokia OVI store}): An electronic healthcare record app maintaining user health data in a portable format for easy understanding by patients and useful reference for doctors.
\end{itemize}

\divider

\cvevent{High School}{Loyola High School}{Class of 2009}{Jamshedpur IN}

\ifthenelse{\boolean{TWO_COLUMN_LAYOUT}} {
\end{paracol}
}{}

\end{document}

